\documentclass{article}
\usepackage[utf8]{inputenc}
\usepackage{graphicx}
\usepackage{geometry}
\geometry{a4paper, total={16cm, 24cm}, top=2cm}
\usepackage{amsmath}
\usepackage{blindtext}
\graphicspath{ {img/} }
\usepackage{listings}
\usepackage{color}
\usepackage{siunitx}
\usepackage{hyperref}
\hypersetup{
    colorlinks=true,
    linkcolor=blue,
    filecolor=magenta,      
    urlcolor=cyan,
    }
\definecolor{dkgreen}{rgb}{0,0.6,0}
\definecolor{gray}{rgb}{0.5,0.5,0.5}
\definecolor{mauve}{rgb}{0.58,0,0.82}

\lstset{ %frame=tb,
  language=C,
  aboveskip=3mm,
  belowskip=3mm,
  showstringspaces=false,
  columns=flexible,
  numbers=left,
  basicstyle={\small\ttfamily},
  numberstyle=\tiny\color{gray},
  keywordstyle=\color{blue},
  commentstyle=\color{dkgreen},
  stringstyle=\color{mauve},
  breaklines=true,
  breakatwhitespace=true,
  tabsize=3  
}


\title{Week 11 Programming Assignment}
\author{Steffen Petersen | au722120}
\date{November 21st 2022}

\begin{document}
%\tableofcontents

\maketitle
\vspace{5pt}
\noindent Here is the link for my repository, in which you will find all the edited code files and such.\\
\url{https://github.com/Aarhus-University-ECE/assignment-11-SirQuacc}
\section{}
\includegraphics[width=\linewidth, keepaspectratio=true]{task1}
\vspace{2pt}\\
We know per definition that the factorial of 1, is 1, and we can confirm that if n=1 in the function, it will pass the assert,
escape the if and enter the else, which returns 1. Thus the base-case of the program, for n=1, must be correct.\\
We also know that other factorials of natural numbers, are the given number n, times all of its previous numbers, until 1, i.e.:\\
$n \cdot (n-1) \cdot (n-2) \cdot ... \cdot 2 \cdot 1$\\
The inductive step of the function is to return $n \cdot fact(n-1)$, meaning if we assume $fact(n-1)$ is correct, multiplying this
by n, will also be correct. And since the base case is correct, we know that n=2 is also correct, and since n=2 is correct n=3 must also be.\\
We could continue this, theoretically, for any positive natural number. This, in combination with seeing that the recursion calls 
the function correctly with $fact(n-1)$, means the whole function is correct. Because the $fact(n-1)$ step ensures we will reach our
base-case of n=1 eventually.


\section{}
\includegraphics[width=\linewidth, keepaspectratio=true]{task2}
\vspace{2pt}
Below is the recursive function, it can also be found in sumn.c
\begin{lstlisting}
  int sumn (int n)
  {
      assert(n >= 1);
      if(n == 1){ // Base case
          return 1; //2*1-1 = 1
      }
      else{
          return 2*n-1 + sumn(n-1); //Recursive step 2*n-1 + 2*(n-1)-1 and so on.
      }
  }
\end{lstlisting}


\section{}
\includegraphics[width=\linewidth, keepaspectratio=true]{task3} 
The code for these can be seen below, and can be found in sum.c
\subsection*{a}
\begin{lstlisting}
  int sumtail (int n, int total)
  {
      assert(n >= 1);
      if(n > 1){
        return sumtail(n-1, n + total); //For n's above 1, update the running total by adding n, and re-call func with n-1
      } else
          return 1 + total; //Base case, when reached, return 1 + running total
  }
\end{lstlisting}
\subsection*{b}
\begin{lstlisting}
  int sumwhile (int n)
  {
    assert(n >= 1);
    int r = 0; //Sum variable to return
    while(n > 0){ //Run loop n times (subtracting 1 each time)
      r+=n; //Add current n value to result variable
      n--;
    }
    return r; 
  }
\end{lstlisting}


\section{}
\includegraphics[width=\linewidth, keepaspectratio=true]{task4}
The tail-recursive function is seen below, using p and pp as running total inputs, starting at the second and first values of the sequence (1 and 0).\\
The function can be found in fib.c
\begin{lstlisting}
  int fib (int n, int p, int pp)
  {
      assert (n >= 1);
      if(n == 1){
          return pp; //If input is simply 1, return pp, should be 0.
      } else if(n == 2){
          return p; //If input is 2 (can happen recursively), return the new p value
      } else 
          return fib(n-1, pp + p, p); //If we want a number in the sequence later than 2, call fib again with updated p and pp as running sums.
  }
\end{lstlisting}


\end{document}